%%======== TEMPLATE INFORMES ESTILO FCFM ========
% Autor: Elisa Paz Kauffmann Figueroa
% Estudiante de la carrera de Ingeniería Civil en Ciencias de la Computación
% de la Faculta de Ciencias Físicas y Matemáticas FCFM de la Universidad de % Chile.
%%===============================================

\documentclass[14pt,letterpaper,hidelinks]{extarticle}
%\usepackage[ansinew]{inputenc}
\usepackage[utf8x]{inputenc}
%\usepackage[latin1]{inputenc}
\usepackage[spanish]{babel}
\usepackage[letterpaper,includeheadfoot, top=1cm, bottom=3.0cm, right=2.0cm, left=2.0cm]{geometry}
\renewcommand{\familydefault}{\sfdefault}

\usepackage{svg}

\usepackage{graphicx}
\usepackage{color}
\usepackage{hyperref}
\usepackage{amssymb}
\usepackage{url}
%\usepackage{pdfpages}
\usepackage{fancyhdr}
\usepackage{hyperref}
\usepackage{subfig}
\usepackage{indentfirst}
\usepackage{titlesec}
\titleformat*{\section}{\large\bfseries}
\titleformat*{\subsection}{\normalsize\bfseries}

\usepackage{listings} %Codigo
\lstset{language=C, tabsize=4,framexleftmargin=5mm,breaklines=true}

\begin{document}
\thispagestyle{empty}
\renewcommand*\listtablename{Índice de tablas}
\renewcommand{\contentsname}{\'Indice}
\renewcommand*{\refname}{}

%\begin{sf}
% --------------- ---------PORTADA --------------------------------------------
\newpage
\pagestyle{fancy}
\fancyhf{}
%-------------------- CABECERA ---------------------
\hbox{\includegraphics[scale=0.3]{img/fcfm_dcc_png.png} }
%------------------ TÍTULO -----------------------
\vspace*{4cm}
\begin{center}
\Huge  {Tarea 2 - Comerciales}\\
\vspace{1cm}
\Large {CC5204 - Búsqueda por contenido de Imágenes y Videos}\\


\vspace{1cm}
\end{center}
%----------------- NOMBRES ------------------------
\begin{flushright}
\begin{table}[h]
	\large
	\centering
	\begin{tabular}{ll}
		Alumnos: & Patricio Isbej\\
				& Elisa Kauffmann\\
		Profesor: & Juan Manuel Barrios\\
		Ayudante: & Sebastián Ferrada Aliaga\\
		Fecha: & 28 de abril de 2016
	\end{tabular}
\end{table}
\end{flushright}

% ·············· ENCABEZADO - PIE DE PAGINA ············
\newpage
\pagestyle{fancy}
\fancyhf{}

%Encabezado
%\fancyhead[L]{\rightmark}
\fancyhead[L]{\small \rm \textit{Sección \nouppercase\rightmark}} %Izquierda
\fancyhead[R]{\small \rm \textbf{\thepage}} %Derecha


%\fancyfoot[L]{\small \rm \textit{Pie de página - Izquierda}} %Izquierda
%\fancyfoot[R]{\small \rm \textit{Pie de página - Derecha}} %Derecha
%\fancyfoot[C]{\thepage} %Centro

\renewcommand{\sectionmark}[1]{\markright{\thesection.\ #1}}
\renewcommand{\headrulewidth}{0.5pt}
%\renewcommand{\footrulewidth}{0.5pt}

% =============== INDICE ===============
%
\tableofcontents
\listoffigures
\listoftables

% =============== SECCIONES Y SUBSECCIONES ===============
\newpage
\section{Resumen}
\section{Hipótesis}
\section{Diseño e implementación}


\textbf{Para ejecutar los tests de cada estructura:}
Para ver el correcto funcionamiento de cada estructuras se implementaron pequeños tests individuales que las construyen y realizan una búsqueda a partir de un texto y una palabra proveídos directamente por el usuario.

A continuación se describen los pasos a seguir para probar cada estructura.

	\begin{itemize}
		\item PatriciaTree y Autómata
		\begin{enumerate}
			\item Para probar el patricia tree, se debe abrir el archivo 			\verb+PatriciaTree.java+ ubicado en el paquete 									\verb+patriciatree+. Para probar el autómata se debe abrir el archivo \verb+PatternSearchAutomaton.java+ ubicado en el paquete \verb+patternsearchautomaton+.

			\item Para especificar el texto sobre el cual se realizará la búsqueda, se debe ir al main al final del archivo y allí, asignar a la variable \verb+text+, entre comillas (`` ''), una frase cuyas palabras estén separadas por espacios. \textbf{NO} se deben incluir espacios ni al comienzo ni al final de la frase.

			\item Para especificar la palabra que se desea buscar, en el mismo main se debe asignar a la variable \verb+word+, entre comillas (`` '') una palabra. \textbf{NO} se debe incluir espacios ni al principio ni al final.
			\item Ejecutar.
		\end{enumerate}

		\item Suffix Tree
			\begin{enumerate}
				\item Abrir el archivo \verb+SuffixTree.java+ ubicado en el paquete \verb+suffixtree+.

				\item Comentar la línea 20 y descomentar la línea 24.

				\item Idéntico al paso 3 descrito en la subsección anterior.

				\item Idéntico al paso 4 descrito en la subsección anterior.
				
				\item Ejecutar.
			\end{enumerate}
\end{itemize}
\newpage

\textbf{Para ejecutar el programa de pruebas sobre libros}
\begin{enumerate}
	\item Abrir el archivo \verb+SuffixTree.java+ ubicado en el paquete \verb+suffixtree+ y asegurarse de que la línea 20 esté descomentada y la línea 24 esté comentada.
	
	\item Abrir el archivo \verb+Main.java+ ubicado en el paquete \verb+main+.

	\item En la línea 24, especificar la ruta del archivo de texto donde se guardarán los resultados.

	\item En la línea 27, especificar la ruta del directorio donde se encuentra la carpeta \verb+libros+ incluído en la carpeta principal de la tarea.
	
	\item Ejecutar.

\end{enumerate}


\section{Experimentos y resultados}

  
\section{Análisis}


%%==================== IMAGENES =====================
%% ················ IMAGEN DOBLE ·················
%\begin{figure}[ht!] \centering
%\subfloat[Logo UChile]{\includegraphics[scale=0.2]{img/escudoU.pdf}}
%\hspace{1cm} %Espacio horizontal
%\subfloat[Logo FCFM]{\includegraphics[scale=0.45]{img/fcfm.png}}
%\caption{Ejemplo de imagen doble}\label{img1}
%\end{figure}
%%··········································
%
%
%A continuación la figura \ref{img2} presenta otra forma de agregar imágenes
%
%% ················ IMAGENES SIMPLES·················
%\begin{figure}[ht!]
%\centering \includegraphics[scale=0.2]{img/escudoU.pdf}
%\caption{Escudo de la Universidad de Chile} \label{img2}
%\end{figure}

%%--------------------------
%
%\begin{figure}[ht!]
%\centering 
%\captionsetup{justification=centering,margin=2cm}
%\includegraphics[scale=0.2]{img/fotos_alma/bunkers.JPG}
%\caption{Burros salvajes junto a los bunkers del campamento donde aloja el personal de OSF} 
%\label{campamento}
%\end{figure}  

%\begin{figure}[ht!]
%\centering \includegraphics[scale=0.2]{img/fotos_alma/cancha_casino.JPG}
%\caption{Multicancha y casino de OSF} 
%\label{cancha}
%\end{figure}  

%----------------------------
%\begin{figure}[ht!]
%\centering
%\hspace*{-2cm}
%\captionsetup{justification=centering,margin=2cm}
%\includegraphics[scale=0.3]{img/figure_2.pdf}
%\caption{Ejemplo de visualización de un perfil de temperatura usando datos de agosto de 2010} 
%\label{surftemp_fig}
%\end{figure}
%\begin{figure}
%\centering
%\hspace*{-2cm}
%\captionsetup{justification=centering,margin=2cm}
%\includegraphics[scale=0.3]{img/figure_3.pdf}
%\caption{Ejemplo de visualización de un perfil de temperatura usando datos de agosto de 2010} 
%\label{intliq_fig}
%\end{figure}
%%··········································



%%============== TABLAS ===============
%\begin{table}[h]
% \centering
% \caption{Headers, identificadores y descripción}
% \begin{tabular}{|| c | c | p{7	cm}||} 
% \hline
% Identificador de header & Tipo de medición & Descripción \\ [0.5ex] 
% \hline\hline
% 10 & - & No se usa (tampoco se describe en el manual)\\
% \hline
% 80 & - & No se usa (tampoco se describe en el manual) \\
% \hline
% 100 & 101 & Tipo de medición y título de los 4 tipos de perfiles\\ 
% \hline
% 200 & 201 & Header para mediciones meteorológicas a nivel de superficie\\
% \hline
% 300 & 301 & Header para mediciones escalares e integradas\\
% \hline
% 400 & 401, 402, 403, 404 & Ángulo de observación y arreglo de valores de 			altura, la varible independientele (58 valores, de 0 a 10 km), para todos 		los perfiles.\\
% \hline
% \end{tabular}
%\end{table}
%\newpage
%
%Las siguientes cuatro filas contienen el tipo de medición ``101", y especifican el tipo de medición para los cuatro tipos de perfiles.
%
%\begin{table}[h]
% \centering
% \caption{Tipo de perfil y datos}
% \begin{tabular}{||c|c|c|c|c||} 
% \hline
% N° de medición & Fecha/Tiempo & 100 &	Tipo de medición & Título \\ [0.5ex] 
% \hline\hline
% 1	& 10/07/14 22:07:14 & 101 & 401 & Temperature(K) \\
% \hline
% 2	& 10/07/14 22:07:14 & 101 & 402 & Vapor Density $(g/m^3)$ \\
% \hline
% 3	& 10/07/14 22:07:14 & 101 & 403 & Liquid Density $(g/m^3)$ \\
% \hline
% 4	& 10/07/14 22:07:14 & 101 & 404 & Relative Humidity (\%) \\
% \hline
% \end{tabular}
%\end{table}
%
%El resto de las filas son mediciones (de tipo 201, 301, 401, 402, 403 y 404). \textbf{Estas son las medidas que van a hacer almacenadas en la base de datos}
%\begin{table}[h]
% \centering
% \caption{Ejemplo de mediciones meteorológicas a nivel de superficie y su correspondiente header}
% \begin{tabular}{||c|c|c|c|c|c|c|c||} 
% \hline
% Record & Date/Time & 200 & Tamb(K) & Rh(\%) & Pres(mb) & Tir(K) & Rain \\ [0.5ex] 
% \hline\hline
% 25 & 10/07/14 22:08:13 & 201 & 276.032 & 10.13 & 557.48 & 191.09 & 0 \\
% \hline
% \end{tabular}
%\end{table}
%
%\newpage
%\begin{table}[h]
% \centering
% \caption{Ejemplo de mediciones de valores scalares e integrados y su correspondiente header}
% \begin{tabular}{||c|c|c|c|c|c|c||} 
% \hline
% Record & Date/Time & 300 &	Int. Vapor(cm) & Int. Liquid(mm) & Cloud Base(km) \\ [0.5ex] 
% \hline\hline
% 42	& 10/07/14 22:08:56 & 301 & 0.077 & 0 & -1 \\
% \hline
% \end{tabular}
%\end{table}
%
%\begin{table}[h]
% \centering
% \caption{Ejemplo de perfiles y sus correspondientes headers}
% \begin{tabular}{||c|c|c|c|c|c|c||} 
% \hline
% Record & Date/Time & 400 &	LV2 Processor(angle) & 0 & ... & 10 \\ [0.5ex] 
% \hline\hline
% 46 & 10/07/14 22:09:41 & 401 &	Zenith & 276.017 & ... & 259.11 \\
% \hline
% 53	& 10/07/14 22:09:43 & 402 &	Angle Scan(N) &	0.623 & ... & 0.007 \\
% \hline
% 57	& 10/07/14 22:09:43 & 403 &	Angle Scan(S) &	0.021 & ... & 0.005 \\ 
% \hline
% 61	& 10/07/14 22:09:43 & 404 &	Angle Scan(A) &	2.805 & ... & 0.336 \\
% \hline
% \end{tabular}
%\end{table}


% ============= REFERENCIAS ==============
%\newpage
%
%\section{Referencias} 
%\begin{thebibliography}{99}
%	\bibitem{nodoPag} R. Jacob Baker, \textit{CMOS. Circuit Design, Layout, and Simulation}, 2nd ed., USA: IEEE Press, 2005
%	\bibitem{Bojan} asdasdas
%	
%\end{thebibliography}


% ============= FIN DE DOCUMENTO ==============
\end{document}
