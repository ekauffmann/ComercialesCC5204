%%======== TEMPLATE INFORMES ESTILO FCFM ========
% Autor: Elisa Paz Kauffmann Figueroa
% Estudiante de Ingeniería Civil en Ciencias de la Computación
% FCFM U. de Chile
%%===============================================

\documentclass[14pt,letterpaper,hidelinks]{extarticle}
%\usepackage[ansinew]{inputenc}
\usepackage[utf8x]{inputenc}
%\usepackage[latin1]{inputenc}
\usepackage[spanish]{babel}
\usepackage[letterpaper,includeheadfoot, top=1cm, bottom=3.0cm, right=2.0cm, left=2.0cm]{geometry}
\renewcommand{\familydefault}{\rmdefault}

\usepackage[table,xcdraw]{xcolor}
\usepackage{svg}
%\usepackage{graphicx}
%\usepackage{color}
\usepackage{hyperref}
\usepackage{amssymb}
\usepackage{url}
%\usepackage{pdfpages}
\usepackage{fancyhdr}
\usepackage{hyperref}
\usepackage{subfig}
\usepackage{indentfirst}
\usepackage{titlesec}
\usepackage{booktabs}
\usepackage{multirow}
\usepackage{listings} %Codigo
\lstset{language=C, tabsize=4,framexleftmargin=5mm,breaklines=true}

\titleformat*{\section}{\large\bfseries}
\titleformat*{\subsection}{\normalsize\bfseries}

\begin{document}
\thispagestyle{empty}
\renewcommand*\listtablename{Índice de tablas}
\renewcommand{\contentsname}{\'Indice}
\renewcommand*{\refname}{}

%\begin{sf}
% --------------- ---------PORTADA --------------------------------------------
\newpage
\pagestyle{fancy}
\fancyhf{}
%-------------------- CABECERA ---------------------
\hbox{\includegraphics[scale=0.3]{img/fcfm_dcc_png.png} }
%------------------ TÍTULO -----------------------
\vspace*{4cm}
\begin{center}
\Huge  {Tarea 2 - Comerciales}\\
\vspace{1cm}
\Large {CC5204 - Búsqueda por contenido de Imágenes y Videos}\\


\vspace{1cm}
\end{center}
%----------------- NOMBRES ------------------------
\begin{flushright}
\begin{table}[h]
	\large
	\centering
	\begin{tabular}{ll}
		Alumnos: & Patricio Isbej\\
				& Elisa Kauffmann\\
		Profesor: & Juan Manuel Barrios\\
		Ayudante: & Sebastián Ferrada\\
		Fecha: & 2 de mayo de 2016
	\end{tabular}
\end{table}
\end{flushright}

% ·············· ENCABEZADO - PIE DE PAGINA ············
\newpage
\pagestyle{fancy}
\fancyhf{}

%Encabezado
%\fancyhead[L]{\rightmark}
\fancyhead[L]{\small \rm \textit{Sección \nouppercase\rightmark}} %Izquierda
\fancyhead[R]{\small \rm \textbf{\thepage}} %Derecha


%\fancyfoot[L]{\small \rm \textit{Pie de página - Izquierda}} %Izquierda
%\fancyfoot[R]{\small \rm \textit{Pie de página - Derecha}} %Derecha
%\fancyfoot[C]{\thepage} %Centro

\renewcommand{\sectionmark}[1]{\markright{\thesection.\ #1}}
\renewcommand{\headrulewidth}{0.5pt}
%\renewcommand{\footrulewidth}{0.5pt}

% =============== INDICE ===============
%
\tableofcontents
\listoffigures
\listoftables

% =============== SECCIONES Y SUBSECCIONES ===============
\newpage
\section{Introducción}
El objetivo de esta tarea es comparar el uso de diferentes descriptores globales para resolver el problema de detección de spots publicitarios (avisos comerciales) en televisión. 
Se dispone de un archivo de video de cuatro horas de duración con la programación emitida por un canal de televisión local, junto con un total de 10 cortos de video, cada uno de largo entre 10 y 40 segundos, correspondientes a spots publicitarios emitidos por el canal durante ese período.
Se pide implementar un detector de comerciales que señale cuántas veces fue emitido cada comercial durante las 4 horas de programación, determinando exactamente el inicio y
fin de cada aparición. Para imlementar el detector de comerciales deberá decidir un método de 
descripción del contenido (extracción de características), un método de búsqueda
por similitud (comparación de descriptores), y un método de detección (encontrar
secuencias parecidas).  
Llamaremos una "configuración" a la elección de extracción de característica,
búsqueda por similitud y método de detección que permiten realizar una detección
de comerciales. Una configuración de ejemplo podría ser: 
\section{Resumen}
\section{Diseño e implementación}
Se implementa un detector de comerciales que señala cuántas veces fue emitido cada comercial durante las 4 horas de programación, determinando exactamente el inicio y
fin de cada aparición. Para ello se divide el trabajo en 3 pasos: 
\begin{enumerate}
\item Descripción del contenido (extracción de características)
\item Búsqueda por similitud (comparación de descriptores)
\item Detección (encontrar secuencias parecidas)
\end{enumerate}

Se usa el lenguaje \verb+Python+ utilizando las librerías \verb+OpenCV+ para procesar imágenes y videos, \verb+numpy+ para manejo de vectores y matrices y \verb+matplotlib+ para graficar.\\

El reporte de detecciones se guarda en un archivo \verb+.m3u+ que por cada aparición de un comercial, contiene 3 líneas con el siguiente formato: 
\begin{verbatim}
#EXTVLCOPT:start-time= [seconds]
#EXTVLCOPT:stop-time= [seconds]
../../base/mega-2014_04_25T22_00_07.mp4
\end{verbatim}
donde la primera y la segunda línea indican el momento en que empieza y termina el comercial respectivamente (en segundos), y la tercera línea indica el video a reproducir, en este caso es el programa de 4 horas.\\

\subsection{Descripción del contenido}
A rasgos generales, en esta etapa se utiliza el descriptor de \textbf{histogramas por zonas} para caracterizar los frames de los videos guardando el descriptor de cada video en un archivo \verb+.npy+, el cual será utilizado en el paso siguiente. Es importante mencionar que los frames de todos los videos son escalados al mismo tamaño ($720\times400$ px) para normalizar y facilitar la división en zonas. Además, previo al desarrollo de la tarea se verificó que todos los videos tuvieran el mismo \textit{framerate} para mantener consistencia ($\sim$ 29.97 fps).\\

En esta etapa, los parámetros que varían en cada configuración son:
\begin{itemize}
\item[-] frecuencia de muestreo de frames
\item[-] cantidad de zonas en las que se divide cada frame
\item[-] cantidad de bins de los histogramas 
\end{itemize}

En detalle, el programa ejecuta los siguientes pasos:
\begin{enumerate}
\item Para cada archivo de video, se toma un frame cada cierta frecuencia, lo escala a un tamaño de $720\times400$ px y lo transforma a escala de grises.
\item Se divide cada uno de estos frames en la cantidad de zonas especificada por el tipo de configuración, y usando la función \verb+calcHist+ de \verb+OpenCV+ se obtiene un histograma por cada zona en forma de un arreglo.
\item Se crea el descriptor completo del frame: un arreglo que es la concatenación de los vectores de los histogramas obtenidos en el paso anterior.
\item Se crea el descriptor del video completo construyendo un arreglo \verb+numpy+ de todos los arreglos que describen cada frame del video.
\end{enumerate}

El código se encuentra en el archivo \verb+describe.py+

\subsection{Búsqueda por similitud}
Durante esta etapa, usando los descriptores creados anteriormente, por cada frame del video de 4 horas se encuentra el frame del comercial que más se le parece. Para ello se comparan todos los vectores descriptores de cada comercial contra cada frame del video de 4 horas, utilizando distancia Manhattan. Para facilitar el procesamiento se crea un diccionario que guarda, por cada comercial, su título, un número identificador y la cantidad de frames procesados. Un ejemplo de uno de los diccionarios creados se pueden observar en el cuadro~\ref{tab:dict}.\\

\begin{table}[]
\centering
\begin{tabular}{@{}ccc@{}}
\toprule
Id & Nombre                      & Frames \\ \midrule
0  & mr big (2).npy              & 95     \\
1  & scotiabank.npy              & 76     \\
2  & ripley dias r.npy           & 63     \\
3  & sodimac homecenter.npy      & 62     \\
4  & donnasept.npy               & 61     \\
5  & dove desodorante hombre.npy & 45     \\
6  & entel 4g corto.npy          & 81     \\
7  & johnson mundial.npy         & 60     \\
8  & limon soda.npy              & 96     \\
9  & bilz y pap.npy              & 105    \\ \bottomrule
\end{tabular}
\captionsetup{justification=centering,margin=2cm}
\caption{Bosquejo del diccionario creado para la configuración de 1 descriptor cada 10 frames. \label{tab:dict}}
\end{table}

Específicamente, el programa realiza los siguientes pasos:
\begin{enumerate}
\item Se crea un diccionario que mapea los títulos de los comerciales a un identificador y a la cantidad de frames obtenidos en la etapa anterior.
\item Se cargan en memoria los descriptores de video de todos los 10 comerciales y usando el diccionario se construye una matriz de dimensiones $10\times 3$. La matriz contiene, para cada comercial, una fila con su identificador, su título y su descriptor. (Ver cuadro~\ref{tab:descsCom}). 
\item Se carga el descriptor del programa de 4 horas y se compara cada frame contra todos los frames de los 10 comerciales. Para medir la similitud entre vectores descriptores se utiliza distancia Manhattan (utilizando la función \verb+norm+ de \verb+OpenCV+ con el parámetro \verb+NORM_L1+). Así se va construyendo un vector de ``calces'' con tantas filas como frames del programa de 4 horas procesados. Cada fila contiene el índice del frame y el identificador del comercial que más se parece al frame del programa.
\item El vector de ``calces'' se guarda en un archivo \verb+.npy+ para ser usado en el paso siguiente   
\end{enumerate} 
\begin{table}[]
\centering
\begin{tabular}{@{}ccc@{}}
\toprule
com\_id & nombre archivo & descriptor comercial \\ \midrule
0 & mr big (2).npy & {[}{[}{]},{[}{]},...,{[}{]},{[}{]}{]} \\
1 & scotiabank.npy & {[}{[}{]},{[}{]},...,{[}{]},{[}{]}{]} \\
2 & ripley dias r.npy & {[}{[}{]},{[}{]},...,{[}{]},{[}{]}{]} \\
3 & sodimac homecenter.npy & {[}{[}{]},{[}{]},...,{[}{]},{[}{]}{]} \\
4 & donnasept.npy & {[}{[}{]},{[}{]},...,{[}{]},{[}{]}{]} \\
5 & dove desodorante hombre.npy & {[}{[}{]},{[}{]},...,{[}{]},{[}{]}{]} \\
6 & entel 4g corto.npy & {[}{[}{]},{[}{]},...,{[}{]},{[}{]}{]} \\
7 & johnson mundial.npy & {[}{[}{]},{[}{]},...,{[}{]},{[}{]}{]} \\
8 & limon soda.npy & {[}{[}{]},{[}{]},...,{[}{]},{[}{]}{]} \\
9 & bilz y pap.npy & {[}{[}{]},{[}{]},...,{[}{]},{[}{]}{]} \\ \bottomrule
\end{tabular}
\captionsetup{justification=centering,margin=2cm}
\caption{Bosquejo de la matriz de identificadores y descriptores de los comerciales \label{tab:descsCom}}
\end{table}

El código se encuentra en el archivo \verb+compare.py+

\subsection{Detección}
En esta etapa se detecta la aparición de los comerciales en el video de 4 horas haciendo uso del diccionario y el vector de ``calces'' creados en las etapas anteriores. La idea general es usar máscaras, una por cada comercial, para ir recorriendo el vector de ``calces'' y comparar la similitud entre segmentos de éste con las máscaras usando distancia de Hamming. La idea se ilustra en la figura~\ref{my-label}.

La distancia de Hamming es calculada usando una función específica implementada para este caso en particular.\\

Detalladamente, la implementación lleva a cabo los siguientes pasos:
\begin{enumerate}
\item Se cargan el diccionario y el vector de ``calces'' en memoria a partir de los archivos generados en la etapa anterior.
\item Para cada comercial en el diccionario se crea una máscara: una matriz con tantas filas como la cantidad de frames del comercial y 2 columnas. En la primera columna se repite para todas las filas el identificador del comercial y en la segunda columna se enumeran todos sus frames. Un ejemplo de máscara se puede ver en la figura~\ref{tab:mask}.
\item Para cada línea del vector de ``calces'' se alínea la máscara del comercial indicado en la posición del frame indicado. Se calcula la similitud entre la máscara y la sección del vector de ``calces'' usando distancia de Hamming. Este proceso se ilustra en la figura~\ref{fig:match}
\item Se reporta una aparición de un comercial cuando la similitud calculada en el paso anterior está dentro de cierto rango de tolerancia. En tal caso no se sigue trabajando con las líneas que corresponden a los frames de comercial, y se coloca el puntero en la posición del último frame de la máscara en el vector de ``calces''.
\item Al encontrar un comercial, se guarda el número del frame inicial y final del calce y se calcula el tiempo de inicio y fin del comercial. Estos datos se guardan en un archivo \verb+.m3u+ para crear la lista de reproducción que reporta las detecciones.
\end{enumerate}

\begin{table}[]
\centering
\begin{tabular}{@{}cc@{}}
\toprule
Id comercial & frame \\ \midrule
2 & 0 \\
2 & 1 \\
2 & 2 \\
2 & 3 \\
2 & 4 \\
2 & 5 \\
2 & 6 \\
2 & 7 \\
2 & 8 \\
2 & 9 \\
2 & 10 \\
2 & 11 \\
2 & 12 \\
2 & 13 \\
2 & 14 \\ \bottomrule
\end{tabular}
\captionsetup{justification=centering,margin=2cm}
\caption{Bosquejo de la máscara de un comercial creada para la configuración de 1 descriptor cada 30 frames\label{tab:mask}}
\end{table}

\begin{figure}[ht!]
\centering 
\captionsetup{justification=centering,margin=2cm}
\includegraphics[scale=0.6]{img/matcheo.png}
\caption{Esquema del proceso de detección de comerciales sobre el vector de ``calces'' usando máscaras.\label{fig:match}}
\end{figure} 

\section{Experimentos y resultados}
\begin{table}[]
\centering
\caption{My caption}
\label{my-label}
\begin{tabular}{c|c|c|c|c|c|c|c|}
\cline{2-8}
 & Frecuencia & \multicolumn{2}{c|}{30} & \multicolumn{2}{c|}{20} & \multicolumn{2}{c|}{10} \\ \hline
\multicolumn{1}{|c|}{Division} & Bins$\backslash$Tolerancia & 50\% & 80\% & 50\% & 80\% & 50\% & 80\% \\ \hline
\multicolumn{1}{|c|}{\multirow{3}{*}{1x1}} & 16 & 0.75 & 0.98 & 0.81 & 1 & 0.84 & 1 \\ \cline{2-8} 
\multicolumn{1}{|c|}{} & 32 & 0.75 & 1 & 0.78 & 1 & 0.89 & 1 \\ \cline{2-8} 
\multicolumn{1}{|c|}{} & 64 & 0.78 & 1 & 0.81 & 1 & 0.89 & 1 \\ \hline
\multicolumn{1}{|c|}{\multirow{3}{*}{2x2}} & 16 & 0.86 & 1 & 0.94 & 1 & 0.94 & 1 \\ \cline{2-8} 
\multicolumn{1}{|c|}{} & 32 & 0.89 & 1 & 0.86 & 1 & 0.94 & 1 \\ \cline{2-8} 
\multicolumn{1}{|c|}{} & 64 & 0.89 & 0.98 & 0.91 & 1 & 0.94 & 1 \\ \hline
\multicolumn{1}{|c|}{\multirow{3}{*}{4x4}} & 16 & 0.91 & 0.96 & 0.89 & 1 & 0.96 & 1 \\ \cline{2-8} 
\multicolumn{1}{|c|}{} & 32 & 0.91 & 0.96 & 0.86 & 1 & 0.94 & 1 \\ \cline{2-8} 
\multicolumn{1}{|c|}{} & 64 & 0.91 & 0.94 & 0.89 & 1 & 0.96 & 1 \\ \hline
\end{tabular}
\end{table}

\begin{table}[]
\centering
\caption{My caption}
\label{my-label}
\begin{tabular}{c|
>{\columncolor[HTML]{C0C0C0}}c |c|c|c|c|c|c|}
\cline{2-8}
                                                                                           & \cellcolor[HTML]{9B9B9B}{\color[HTML]{000000} Frecuencia} & \multicolumn{2}{c|}{\cellcolor[HTML]{9B9B9B}{\color[HTML]{000000} 30}} & \multicolumn{2}{c|}{\cellcolor[HTML]{9B9B9B}{\color[HTML]{000000} 20}} & \multicolumn{2}{c|}{\cellcolor[HTML]{9B9B9B}{\color[HTML]{000000} 10}} \\ \hline
\multicolumn{1}{|c|}{\cellcolor[HTML]{9B9B9B}{\color[HTML]{000000} Division}}              & Bins$\backslash$Tolerancia                                           & \cellcolor[HTML]{C0C0C0}50\%       & \cellcolor[HTML]{C0C0C0}80\%      & \cellcolor[HTML]{C0C0C0}50\%       & \cellcolor[HTML]{C0C0C0}80\%      & \cellcolor[HTML]{C0C0C0}50\%       & \cellcolor[HTML]{C0C0C0}80\%      \\ \hline
\multicolumn{1}{|c|}{\cellcolor[HTML]{9B9B9B}{\color[HTML]{000000} }}                      & {\color[HTML]{000000} 16}                                 & 15/0/10                            & 24/0/1                            & 17/0/8                             & 25/0/0                            & 18/0/7                             & 25/0/0                            \\ \cline{2-8} 
\multicolumn{1}{|c|}{\cellcolor[HTML]{9B9B9B}{\color[HTML]{000000} }}                      & {\color[HTML]{000000} 32}                                 & 15/0/10                            & 25/0/0                            & 16/0/9                             & 25/0/0                            & 20/0/5                             & 25/0/0                            \\ \cline{2-8} 
\multicolumn{1}{|c|}{\multirow{-3}{*}{\cellcolor[HTML]{9B9B9B}{\color[HTML]{000000} 1x1}}} & {\color[HTML]{000000} 64}                                 & 16/0/9                             & 25/0/0                            & 20/0/5                             & 25/0/0                            & 20/0/5                             & 25/0/0                            \\ \hline
\multicolumn{1}{|c|}{\cellcolor[HTML]{9B9B9B}{\color[HTML]{000000} }}                      & {\color[HTML]{000000} 16}                                 & 19/0/6                             & 25/0/0                            & 22/0/3                             & 25/0/0                            & 22/0/3                             & 25/0/0                            \\ \cline{2-8} 
\multicolumn{1}{|c|}{\cellcolor[HTML]{9B9B9B}{\color[HTML]{000000} }}                      & {\color[HTML]{000000} 32}                                 & 20/0/5                             & 25/0/0                            & 19/0/6                             & 25/0/0                            & 22/0/3                             & 25/0/0                            \\ \cline{2-8} 
\multicolumn{1}{|c|}{\multirow{-3}{*}{\cellcolor[HTML]{9B9B9B}{\color[HTML]{000000} 2x2}}} & {\color[HTML]{000000} 64}                                 & 20/0/5                             & 25/1/0                            & 21/0/4                             & 25/0/0                            & 22/0/3                             & 25/0/0                            \\ \hline
\multicolumn{1}{|c|}{\cellcolor[HTML]{9B9B9B}{\color[HTML]{000000} }}                      & {\color[HTML]{000000} 16}                                 & 21/0/4                             & 25/2/0                            & 20/0/5                             & 25/0/0                            & 23/0/2                             & 25/0/0                            \\ \cline{2-8}
\multicolumn{1}{|c|}{\cellcolor[HTML]{9B9B9B}{\color[HTML]{000000} }}                      & {\color[HTML]{000000} 32}                                 & 21/0/4                             & 25/2/0                            & 19/0/6                             & 25/0/0                            & 22/0/3                             & 25/0/0                            \\ \cline{2-8} 
\multicolumn{1}{|c|}{\multirow{-3}{*}{\cellcolor[HTML]{9B9B9B}{\color[HTML]{000000} 4x4}}} & {\color[HTML]{000000} 64}                                 & 21/0/4                             & 25/3/0                            & 20/0/5                             & 25/0/0                            & 23/0/2                             & 25/0/0                            \\ \hline
\end{tabular}
\end{table}
\section{Análisis y conclusiones}
\section{Anexos}
\begin{table}[]
\centering
\caption{My caption}
\label{my-label}
\begin{tabular}{|cc|l|l|l|l|l|l|}
\hline
\multicolumn{1}{|l}{} & \multicolumn{1}{r|}{Frecuencia} & \multicolumn{2}{c|}{30}                               & \multicolumn{2}{c|}{20}                               & \multicolumn{2}{c|}{10}                               \\ \cline{3-8} 
\multicolumn{1}{|l}{Division} & \multicolumn{1}{l|}{Bins$\backslash$Tolerancia} & \multicolumn{1}{c|}{50\%} & \multicolumn{1}{c|}{80\%} & \multicolumn{1}{c|}{50\%} & \multicolumn{1}{c|}{80\%} & \multicolumn{1}{c|}{50\%} & \multicolumn{1}{c|}{80\%} \\ \hline
\multicolumn{1}{|c|}{\multirow{3}{*}{1x1}} & 16                                   & 15/0/10                   & 24/0/1                    & 17/0/8                    & 25/0/0                    & 18/0/7                    & 25/0/0                    \\ \cline{2-8} 
\multicolumn{1}{|c|}{}                     & 32                                   & 15/0/10                   & 25/0/0                    & 16/0/9                    & 25/0/0                    & 20/0/5                    & 25/0/0                    \\ \cline{2-8} 
\multicolumn{1}{|c|}{}                     & 64                                   & 16/0/9                    & 25/0/0                    & 20/0/5                    & 25/0/0                    & 20/0/5                    & 25/0/0                    \\ \hline
\multicolumn{1}{|c|}{\multirow{3}{*}{2x2}} & 16                                   & 19/0/6                    & 25/0/0                    & 22/0/3                    & 25/0/0                    & 22/0/3                    & 25/0/0                    \\ \cline{2-8} 
\multicolumn{1}{|c|}{}                     & 32                                   & 20/0/5                    & 25/0/0                    & 19/0/6                    & 25/0/0                    & 22/0/3                    & 25/0/0                    \\ \cline{2-8} 
\multicolumn{1}{|c|}{}                     & 64                                   & 20/0/5                    & 25/1/0                    & 21/0/4                    & 25/0/0                    & 22/0/3                    & 25/0/0                    \\ \hline
\multicolumn{1}{|c|}{\multirow{3}{*}{4x4}} & 16                                   & 21/0/4                    & 25/2/0                    & 20/0/5                    & 25/0/0                    & 23/0/2                    & 25/0/0                    \\ \cline{2-8} 
\multicolumn{1}{|c|}{}                     & 32                                   & 21/0/4                    & 25/2/0                    & 19/0/6                    & 25/0/0                    & 22/0/3                    & 25/0/0                    \\ \cline{2-8} 
\multicolumn{1}{|c|}{}                     & 64                                   & 21/0/4                    & 25/3/0                    & 20/0/5                    & 25/0/0                    & 23/0/2                    & 25/0/0                    \\ \hline
\end{tabular}
\end{table}

\begin{table}[]
\centering
\caption{My caption}
\label{my-label}
\begin{tabular}{ll}
2 & 0  \\
2 & 1  \\
2 & 2  \\
2 & 3  \\
2 & 4  \\
2 & 5  \\
2 & 6  \\
2 & 7  \\
2 & 8  \\
2 & 9  \\
2 & 10 \\
2 & 11 \\
2 & 12 \\
2 & 13 \\
2 & 14
\end{tabular}
\end{table}
\begin{table}[]
\centering
\caption{My caption}
\label{my-label}
\begin{tabular}{ll}
8 & 9  \\
4 & 16 \\
5 & 29 \\
5 & 29 \\
5 & 29 \\
4 & 16 \\
4 & 16 \\
2 & 1  \\
2 & 5  \\
2 & 3  \\
2 & 3  \\
2 & 5  \\
2 & 5  \\
2 & 7  \\
2 & 7  \\
2 & 8  \\
2 & 9  \\
2 & 11 \\
2 & 11 \\
2 & 12 \\
2 & 14 \\
2 & 14 \\
4 & 6  \\
4 & 6  \\
4 & 6  \\
4 & 6  \\
6 & 23 \\
4 & 15 \\
4 & 7  \\
5 & 25 \\
2 & 0  \\
4 & 3 
\end{tabular}
\end{table}

alineamiento\\

máscara\\

frame actual\\

calce\\

tamaño máscara\\
%%==================== IMAGENES =====================
%% ················ IMAGEN DOBLE ·················
%\begin{figure}[ht!] \centering
%\subfloat[Logo UChile]{\includegraphics[scale=0.2]{img/escudoU.pdf}}
%\hspace{1cm} %Espacio horizontal
%\subfloat[Logo FCFM]{\includegraphics[scale=0.45]{img/fcfm.png}}
%\caption{Ejemplo de imagen doble}\label{img1}
%\end{figure}
%%··········································
%
%
%A continuación la figura \ref{img2} presenta otra forma de agregar imágenes
%
%% ················ IMAGENES SIMPLES·················
%\begin{figure}[ht!]
%\centering \includegraphics[scale=0.2]{img/escudoU.pdf}
%\caption{Escudo de la Universidad de Chile} \label{img2}
%\end{figure}

%%--------------------------
%
%\begin{figure}[ht!]
%\centering 
%\captionsetup{justification=centering,margin=2cm}
%\includegraphics[scale=0.2]{img/fotos_alma/bunkers.JPG}
%\caption{Burros salvajes junto a los bunkers del campamento donde aloja el personal de OSF} 
%\label{campamento}
%\end{figure}  

%\begin{figure}[ht!]
%\centering \includegraphics[scale=0.2]{img/fotos_alma/cancha_casino.JPG}
%\caption{Multicancha y casino de OSF} 
%\label{cancha}
%\end{figure}  

%----------------------------
%\begin{figure}[ht!]
%\centering
%\hspace*{-2cm}
%\captionsetup{justification=centering,margin=2cm}
%\includegraphics[scale=0.3]{img/figure_2.pdf}
%\caption{Ejemplo de visualización de un perfil de temperatura usando datos de agosto de 2010} 
%\label{surftemp_fig}
%\end{figure}
%\begin{figure}
%\centering
%\hspace*{-2cm}
%\captionsetup{justification=centering,margin=2cm}
%\includegraphics[scale=0.3]{img/figure_3.pdf}
%\caption{Ejemplo de visualización de un perfil de temperatura usando datos de agosto de 2010} 
%\label{intliq_fig}
%\end{figure}
%%··········································



%%============== TABLAS ===============
%\begin{table}[h]
% \centering
% \caption{Headers, identificadores y descripción}
% \begin{tabular}{|| c | c | p{7	cm}||} 
% \hline
% Identificador de header & Tipo de medición & Descripción \\ [0.5ex] 
% \hline\hline
% 10 & - & No se usa (tampoco se describe en el manual)\\
% \hline
% 80 & - & No se usa (tampoco se describe en el manual) \\
% \hline
% 100 & 101 & Tipo de medición y título de los 4 tipos de perfiles\\ 
% \hline
% 200 & 201 & Header para mediciones meteorológicas a nivel de superficie\\
% \hline
% 300 & 301 & Header para mediciones escalares e integradas\\
% \hline
% 400 & 401, 402, 403, 404 & Ángulo de observación y arreglo de valores de 			altura, la varible independientele (58 valores, de 0 a 10 km), para todos 		los perfiles.\\
% \hline
% \end{tabular}
%\end{table}
%\newpage
%
%Las siguientes cuatro filas contienen el tipo de medición ``101", y especifican el tipo de medición para los cuatro tipos de perfiles.
%
%\begin{table}[h]
% \centering
% \caption{Tipo de perfil y datos}
% \begin{tabular}{||c|c|c|c|c||} 
% \hline
% N° de medición & Fecha/Tiempo & 100 &	Tipo de medición & Título \\ [0.5ex] 
% \hline\hline
% 1	& 10/07/14 22:07:14 & 101 & 401 & Temperature(K) \\
% \hline
% 2	& 10/07/14 22:07:14 & 101 & 402 & Vapor Density $(g/m^3)$ \\
% \hline
% 3	& 10/07/14 22:07:14 & 101 & 403 & Liquid Density $(g/m^3)$ \\
% \hline
% 4	& 10/07/14 22:07:14 & 101 & 404 & Relative Humidity (\%) \\
% \hline
% \end{tabular}
%\end{table}
%
%El resto de las filas son mediciones (de tipo 201, 301, 401, 402, 403 y 404). \textbf{Estas son las medidas que van a hacer almacenadas en la base de datos}
%\begin{table}[h]
% \centering
% \caption{Ejemplo de mediciones meteorológicas a nivel de superficie y su correspondiente header}
% \begin{tabular}{||c|c|c|c|c|c|c|c||} 
% \hline
% Record & Date/Time & 200 & Tamb(K) & Rh(\%) & Pres(mb) & Tir(K) & Rain \\ [0.5ex] 
% \hline\hline
% 25 & 10/07/14 22:08:13 & 201 & 276.032 & 10.13 & 557.48 & 191.09 & 0 \\
% \hline
% \end{tabular}
%\end{table}
%
%\newpage
%\begin{table}[h]
% \centering
% \caption{Ejemplo de mediciones de valores scalares e integrados y su correspondiente header}
% \begin{tabular}{||c|c|c|c|c|c|c||} 
% \hline
% Record & Date/Time & 300 &	Int. Vapor(cm) & Int. Liquid(mm) & Cloud Base(km) \\ [0.5ex] 
% \hline\hline
% 42	& 10/07/14 22:08:56 & 301 & 0.077 & 0 & -1 \\
% \hline
% \end{tabular}
%\end{table}
%
%\begin{table}[h]
% \centering
% \caption{Ejemplo de perfiles y sus correspondientes headers}
% \begin{tabular}{||c|c|c|c|c|c|c||} 
% \hline
% Record & Date/Time & 400 &	LV2 Processor(angle) & 0 & ... & 10 \\ [0.5ex] 
% \hline\hline
% 46 & 10/07/14 22:09:41 & 401 &	Zenith & 276.017 & ... & 259.11 \\
% \hline
% 53	& 10/07/14 22:09:43 & 402 &	Angle Scan(N) &	0.623 & ... & 0.007 \\
% \hline
% 57	& 10/07/14 22:09:43 & 403 &	Angle Scan(S) &	0.021 & ... & 0.005 \\ 
% \hline
% 61	& 10/07/14 22:09:43 & 404 &	Angle Scan(A) &	2.805 & ... & 0.336 \\
% \hline
% \end{tabular}
%\end{table}


% ============= REFERENCIAS ==============
%\newpage
%
%\section{Referencias} 
%\begin{thebibliography}{99}
%	\bibitem{nodoPag} R. Jacob Baker, \textit{CMOS. Circuit Design, Layout, and Simulation}, 2nd ed., USA: IEEE Press, 2005
%	\bibitem{Bojan} asdasdas
%	
%\end{thebibliography}


% ============= FIN DE DOCUMENTO ==============
\end{document}
