%%======== TEMPLATE INFORMES ESTILO FCFM ========
% Autor: Elisa Paz Kauffmann Figueroa
% Estudiante de Ingeniería Civil en Ciencias de la Computación
% FCFM U. de Chile
%%===============================================

\documentclass[14pt,letterpaper,hidelinks]{extarticle}
%\usepackage[ansinew]{inputenc}
\usepackage[utf8x]{inputenc}
%\usepackage[latin1]{inputenc}
\usepackage[spanish]{babel}
\usepackage[letterpaper,includeheadfoot, top=1cm, bottom=3.0cm, right=2.0cm, left=2.0cm]{geometry}
\renewcommand{\familydefault}{\rmdefault}

\usepackage{svg}

\usepackage{graphicx}
\usepackage{color}
\usepackage{hyperref}
\usepackage{amssymb}
\usepackage{url}
%\usepackage{pdfpages}
\usepackage{fancyhdr}
\usepackage{hyperref}
\usepackage{subfig}
\usepackage{indentfirst}
\usepackage{titlesec}

\titleformat*{\section}{\large\bfseries}
\titleformat*{\subsection}{\normalsize\bfseries}

\usepackage{listings} %Codigo
\lstset{language=C, tabsize=4,framexleftmargin=5mm,breaklines=true}

\begin{document}
\thispagestyle{empty}
\renewcommand*\listtablename{Índice de tablas}
\renewcommand{\contentsname}{\'Indice}
\renewcommand*{\refname}{}

%\begin{sf}
% --------------- ---------PORTADA --------------------------------------------
\newpage
\pagestyle{fancy}
\fancyhf{}
%-------------------- CABECERA ---------------------
\hbox{\includegraphics[scale=0.3]{img/fcfm_dcc_png.png} }
%------------------ TÍTULO -----------------------
\vspace*{4cm}
\begin{center}
\Huge  {Tarea 2 - Comerciales}\\
\vspace{1cm}
\Large {CC5204 - Búsqueda por contenido de Imágenes y Videos}\\


\vspace{1cm}
\end{center}
%----------------- NOMBRES ------------------------
\begin{flushright}
\begin{table}[h]
	\large
	\centering
	\begin{tabular}{ll}
		Alumnos: & Patricio Isbej\\
				& Elisa Kauffmann\\
		Profesor: & Juan Manuel Barrios\\
		Ayudante: & Sebastián Ferrada\\
		Fecha: & 2 de mayo de 2016
	\end{tabular}
\end{table}
\end{flushright}

% ·············· ENCABEZADO - PIE DE PAGINA ············
\newpage
\pagestyle{fancy}
\fancyhf{}

%Encabezado
%\fancyhead[L]{\rightmark}
\fancyhead[L]{\small \rm \textit{Sección \nouppercase\rightmark}} %Izquierda
\fancyhead[R]{\small \rm \textbf{\thepage}} %Derecha


%\fancyfoot[L]{\small \rm \textit{Pie de página - Izquierda}} %Izquierda
%\fancyfoot[R]{\small \rm \textit{Pie de página - Derecha}} %Derecha
%\fancyfoot[C]{\thepage} %Centro

\renewcommand{\sectionmark}[1]{\markright{\thesection.\ #1}}
\renewcommand{\headrulewidth}{0.5pt}
%\renewcommand{\footrulewidth}{0.5pt}

% =============== INDICE ===============
%
\tableofcontents
\listoffigures
\listoftables

% =============== SECCIONES Y SUBSECCIONES ===============
\newpage
\section{Introducción}
\section{Resumen}
\section{Hipótesis}
\section{Diseño e implementación}
Se implementa un detector de comerciales que señala cuántas veces fue emitido cada comercial durante las 4 horas de programación, determinando exactamente el inicio y
fin de cada aparición. El reporte de detecciones se guarda en un archivo \verb+.m3u+ que por cada aparición de un comercial, contiene 3 líneas con el siguiente formato: 
\begin{verbatim}
#EXTVLCOPT:start-time= [seconds]
#EXTVLCOPT:stop-time= [seconds]
../../base/mega-2014_04_25T22_00_07.mp4
\end{verbatim}
donde la primera y la segunda línea indican el momento en que empieza y termina el comercial respectivamente (en segundos), y la tercera línea indica el video a reproducir, en este caso es el programa de 4 horas.\\

Para implementar el detector de comerciales se divide el trabajo en 3 pasos: 
\begin{enumerate}
\item Descripción del contenido (extracción de características)
\item Búsqueda por similitud (comparación de descriptores)
\item Detección (encontrar secuencias parecidas)
\end{enumerate}

\subsection{Descripción del contenido}
A rasgos generales, en esta etapa se utiliza el descriptor de \textbf{histogramas por zonas} para caracterizar los frames de los videos guardando el descriptor de cada video en un archivo \verb+.npy+, el cual será utilizado en el paso siguiente. Es importante mencionar que los frames de todos los videos son escalados al mismo tamaño ($720\times400$ px) para normalizar y facilitar la división en zonas. Además, previo al desarrollo de la tarea se verificó que todos los videos tuvieran el mismo \textit{framerate} para mantener consistencia ($\sim$ 29.97 fps).\\

En esta etapa, los parámetros que varían en cada configuración son:
\begin{itemize}
\item[-] frecuencia de muestreo de frames
\item[-] cantidad de zonas en las que se divide cada frame
\item[-] cantidad de bins de los histogramas 
\end{itemize}

En detalle, el programa ejecuta los siguientes pasos:
\begin{enumerate}
\item Para cada archivo de video, se toma un frame cada cierta frecuencia, lo escala a un tamaño de $720\times400$ px y lo transforma a escala de grises.
\item Se divide cada uno de estos frames en la cantidad de zonas especificada por el tipo de configuración, y usando la función \verb+calcHist+ de \verb+OpenCV+ se obtiene un histograma por cada zona en forma de un arreglo \verb+numpy+.
\item Se crea el descriptor completo del frame: un arreglo \verb+numpy+ que es la concatenación de los vectores de los histogramas obtenidos en el paso anterior.
\item Se crea el descriptor del video completo construyendo un arreglo \verb+numpy+ de todos los arreglos que describen cada frame del video.
\end{enumerate}
\subsection{Búsqueda por similitud}
\subsection{Detección}
 
\section{Experimentos y resultados}
\section{Análisis y conclusiones}


%%==================== IMAGENES =====================
%% ················ IMAGEN DOBLE ·················
%\begin{figure}[ht!] \centering
%\subfloat[Logo UChile]{\includegraphics[scale=0.2]{img/escudoU.pdf}}
%\hspace{1cm} %Espacio horizontal
%\subfloat[Logo FCFM]{\includegraphics[scale=0.45]{img/fcfm.png}}
%\caption{Ejemplo de imagen doble}\label{img1}
%\end{figure}
%%··········································
%
%
%A continuación la figura \ref{img2} presenta otra forma de agregar imágenes
%
%% ················ IMAGENES SIMPLES·················
%\begin{figure}[ht!]
%\centering \includegraphics[scale=0.2]{img/escudoU.pdf}
%\caption{Escudo de la Universidad de Chile} \label{img2}
%\end{figure}

%%--------------------------
%
%\begin{figure}[ht!]
%\centering 
%\captionsetup{justification=centering,margin=2cm}
%\includegraphics[scale=0.2]{img/fotos_alma/bunkers.JPG}
%\caption{Burros salvajes junto a los bunkers del campamento donde aloja el personal de OSF} 
%\label{campamento}
%\end{figure}  

%\begin{figure}[ht!]
%\centering \includegraphics[scale=0.2]{img/fotos_alma/cancha_casino.JPG}
%\caption{Multicancha y casino de OSF} 
%\label{cancha}
%\end{figure}  

%----------------------------
%\begin{figure}[ht!]
%\centering
%\hspace*{-2cm}
%\captionsetup{justification=centering,margin=2cm}
%\includegraphics[scale=0.3]{img/figure_2.pdf}
%\caption{Ejemplo de visualización de un perfil de temperatura usando datos de agosto de 2010} 
%\label{surftemp_fig}
%\end{figure}
%\begin{figure}
%\centering
%\hspace*{-2cm}
%\captionsetup{justification=centering,margin=2cm}
%\includegraphics[scale=0.3]{img/figure_3.pdf}
%\caption{Ejemplo de visualización de un perfil de temperatura usando datos de agosto de 2010} 
%\label{intliq_fig}
%\end{figure}
%%··········································



%%============== TABLAS ===============
%\begin{table}[h]
% \centering
% \caption{Headers, identificadores y descripción}
% \begin{tabular}{|| c | c | p{7	cm}||} 
% \hline
% Identificador de header & Tipo de medición & Descripción \\ [0.5ex] 
% \hline\hline
% 10 & - & No se usa (tampoco se describe en el manual)\\
% \hline
% 80 & - & No se usa (tampoco se describe en el manual) \\
% \hline
% 100 & 101 & Tipo de medición y título de los 4 tipos de perfiles\\ 
% \hline
% 200 & 201 & Header para mediciones meteorológicas a nivel de superficie\\
% \hline
% 300 & 301 & Header para mediciones escalares e integradas\\
% \hline
% 400 & 401, 402, 403, 404 & Ángulo de observación y arreglo de valores de 			altura, la varible independientele (58 valores, de 0 a 10 km), para todos 		los perfiles.\\
% \hline
% \end{tabular}
%\end{table}
%\newpage
%
%Las siguientes cuatro filas contienen el tipo de medición ``101", y especifican el tipo de medición para los cuatro tipos de perfiles.
%
%\begin{table}[h]
% \centering
% \caption{Tipo de perfil y datos}
% \begin{tabular}{||c|c|c|c|c||} 
% \hline
% N° de medición & Fecha/Tiempo & 100 &	Tipo de medición & Título \\ [0.5ex] 
% \hline\hline
% 1	& 10/07/14 22:07:14 & 101 & 401 & Temperature(K) \\
% \hline
% 2	& 10/07/14 22:07:14 & 101 & 402 & Vapor Density $(g/m^3)$ \\
% \hline
% 3	& 10/07/14 22:07:14 & 101 & 403 & Liquid Density $(g/m^3)$ \\
% \hline
% 4	& 10/07/14 22:07:14 & 101 & 404 & Relative Humidity (\%) \\
% \hline
% \end{tabular}
%\end{table}
%
%El resto de las filas son mediciones (de tipo 201, 301, 401, 402, 403 y 404). \textbf{Estas son las medidas que van a hacer almacenadas en la base de datos}
%\begin{table}[h]
% \centering
% \caption{Ejemplo de mediciones meteorológicas a nivel de superficie y su correspondiente header}
% \begin{tabular}{||c|c|c|c|c|c|c|c||} 
% \hline
% Record & Date/Time & 200 & Tamb(K) & Rh(\%) & Pres(mb) & Tir(K) & Rain \\ [0.5ex] 
% \hline\hline
% 25 & 10/07/14 22:08:13 & 201 & 276.032 & 10.13 & 557.48 & 191.09 & 0 \\
% \hline
% \end{tabular}
%\end{table}
%
%\newpage
%\begin{table}[h]
% \centering
% \caption{Ejemplo de mediciones de valores scalares e integrados y su correspondiente header}
% \begin{tabular}{||c|c|c|c|c|c|c||} 
% \hline
% Record & Date/Time & 300 &	Int. Vapor(cm) & Int. Liquid(mm) & Cloud Base(km) \\ [0.5ex] 
% \hline\hline
% 42	& 10/07/14 22:08:56 & 301 & 0.077 & 0 & -1 \\
% \hline
% \end{tabular}
%\end{table}
%
%\begin{table}[h]
% \centering
% \caption{Ejemplo de perfiles y sus correspondientes headers}
% \begin{tabular}{||c|c|c|c|c|c|c||} 
% \hline
% Record & Date/Time & 400 &	LV2 Processor(angle) & 0 & ... & 10 \\ [0.5ex] 
% \hline\hline
% 46 & 10/07/14 22:09:41 & 401 &	Zenith & 276.017 & ... & 259.11 \\
% \hline
% 53	& 10/07/14 22:09:43 & 402 &	Angle Scan(N) &	0.623 & ... & 0.007 \\
% \hline
% 57	& 10/07/14 22:09:43 & 403 &	Angle Scan(S) &	0.021 & ... & 0.005 \\ 
% \hline
% 61	& 10/07/14 22:09:43 & 404 &	Angle Scan(A) &	2.805 & ... & 0.336 \\
% \hline
% \end{tabular}
%\end{table}


% ============= REFERENCIAS ==============
%\newpage
%
%\section{Referencias} 
%\begin{thebibliography}{99}
%	\bibitem{nodoPag} R. Jacob Baker, \textit{CMOS. Circuit Design, Layout, and Simulation}, 2nd ed., USA: IEEE Press, 2005
%	\bibitem{Bojan} asdasdas
%	
%\end{thebibliography}


% ============= FIN DE DOCUMENTO ==============
\end{document}
